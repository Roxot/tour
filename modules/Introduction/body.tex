\begin{frame}{This tutorial is about}

	{\bf probabilistic} models \emph{parameterised} by {\bf neural networks}
	\begin{itemize}
		\item in this class we will look into what this means, why this is interesting, and when this is challenging
	\end{itemize}
	
\end{frame}

\frame{\tableofcontents}

\section{Probabilistic Models}


\begin{frame}{What is a probabilistic model?}

	A probabilistic model predicts possible outcomes of an experiment. 
	
	~
	
	Most modern machine learning models are probabilistic.	

\end{frame}

\begin{frame}{Two Machine Learning Paradigms}

	Supervised problems: \alert{learn a distribution over observed data}
	\begin{itemize}
		\item sentences in natural language, images, videos, $\ldots$
	\end{itemize}
	
	~
	
	Unsupervised problems: \alert{learn a distribution over observed and unobserved data}
	\begin{itemize}
		\item sentences in natural language + parse trees, images + bounding boxes, $\ldots$
	\end{itemize}
\end{frame}

\begin{frame}{What are the benefits of probabilistic models?}

	Probabilistic models allows to incorporate assumptions through
	\begin{itemize}
		\item the choice of distribution
		\item the way that distributions uses side information
		\item stipulate unobserved data and their properties
	\end{itemize}
	
	~
	
	They return a distribution over outcomes	
	
\end{frame}

\begin{frame}{Other benefits}

	\begin{itemize}
		\item They can generate data
		\item They allow to model unobserved data
		\item They can be more compact
		\item They can provide explanation and can suggest improvements
		\item They can inform decision makers
	\end{itemize}
	
\end{frame}



\begin{frame}{Deep Generative Models}

	Naturally, one would like to combine the advantages of probabilistic models and neural networks. So why not have a neural net with latent variables?
	
	~ \pause
	
	Short answer: \emph{backpropagation breaks!}
\end{frame}

\begin{frame}{Why are we here today?}

	Because we want to combine the advantages of probabilistic models and neural networks to potentially 
	\begin{itemize}
		\item overcome lack of supervision
		\item learn from partial supervision
		\item learn from less data
		\item shape the way models reason about data
	\end{itemize}
	and much more!
	
\end{frame}

\begin{frame}{What are you getting out of this today?}

	As we progress we will
	\begin{itemize}
		\item develop a shared vocabulary to talk about probabilistic models powered by NNs
		\item derive crucial results step by step
		\item connect concepts and implementation
	\end{itemize}

	\pause

	Goal
	\begin{itemize}
		\item you should be able to navigate through fresh literature
		\item and start combining probabilistic models and NNs
	\end{itemize}

\end{frame}

\section{Supervised Models Powered by NNs}
\frame{\tableofcontents[currentsection]}

\begin{frame}{Supervised problems}

We have data $x^{(1)}, \ldots,  x^{(N)}$ \textcolor{gray}{~ e.g. sentences, images} \\
generated by some {\bf unknown} procedure 
which we assume can be captured by a probabilistic model


\begin{itemize}
	\item with {\bf known} probability (mass/density) function e.g.
	\begin{align*}
    X \sim \Cat(\alert{\theta_1}, \alert{\ldots}, \alert{\theta_K}) & & \text{or} & & X \sim \mathcal N(\alert{\theta_\mu}, \alert{\theta^2_\sigma})
    \end{align*}    
\end{itemize}
\pause
and \alert{estimate parameters $\theta$}  that assign maximum likelihood $p(x^{(1)}, \ldots, x^{(N)}|\theta)$ to observations

\end{frame}


\begin{frame}{Supervised NN models}

Let $y$ be all side information available\\
~ e.g. deterministic \emph{inputs/features/predictors}

~

Have neural networks predict parameters of our probabilistic model
	\begin{align*}
    X|y \sim \Cat(\pi_{\alert \theta}(y)) & & \text{or} & & X|y \sim \mathcal N(\mu_{\alert \theta}(y), \sigma_{\alert \theta}(y)^2)
    \end{align*}
~ and proceed to \alert{estimate parameters $\theta$}  of the NNs %via MLE % that assign maximum likelihood to observations



\end{frame}



\begin{frame}{Graphical model}

\begin{columns}
\begin{column}{0.6\textwidth}

Random variables
\begin{itemize}
	\item observed data \\
	$x^{(1)}, \ldots, x^{(N)}$
\end{itemize}

Deterministic variables
\begin{itemize}
	\item inputs or predictors \\
	$y^{(1)}, \ldots, y^{(N)}$
	\item model parameters $\theta$
\end{itemize}

\end{column}
\begin{column}{0.2\textwidth}
\begin{tikzpicture}
\node[obs] (x) {$ x $};
\node[above=of x] (y) {$ y $};
\node[right=of x] (theta) {$ \theta $};
\edge{y,theta}{x};

\plate {data} {(y) (x)} {$ N $};
\end{tikzpicture}
\end{column}

\end{columns}


\end{frame}


\begin{frame}{Multiple problems, same language}



\begin{small}

\begin{columns}
\begin{column}{0.3\textwidth}
\scalebox{0.8}{
\begin{tikzpicture}
\node[obs] (x) {$ x $};
\node[left=of x] (y) {$ y $};
\node[right=of x](theta) {$\theta$};
\edge{theta,y}{x};

\plate {data} {(y)(x)} {$ N $};
\end{tikzpicture}
}
\end{column}
\begin{column}{0.6\textwidth}
\alert{(Conditional) Density estimation}
\end{column}

\end{columns}

\begin{tabular}{p{2cm} p{4cm} p{4cm}}
 & Side information ($y$) & Observation ($x$) \\
Parsing &   \textcolor{gray}{a sentence} & \textcolor{black}{its syntactic/semantic parse tree/graph} \\
&&\\
Translation &  \textcolor{gray}{a sentence} & \textcolor{black}{its translation} \\
&&\\
Captioning &  \textcolor{gray}{an image} & \textcolor{black}{caption in English} \\
&&\\
Entailment  & \textcolor{gray}{a text and hypothesis} & \textcolor{black}{entailment relation}
\end{tabular}
\end{small}

\end{frame}


\begin{frame}{Task-driven feature extraction}

Often our side information is itself some high dimensional object
\begin{itemize}
	\item $y$ is a sentence and $x$ a tree
	\item $y$ is the source sentence and $x$ is the target
	\item $y$ is an image and $x$ is a caption
\end{itemize}
and part of the job of the NNs that parametrise our models is to also \alert{deterministically} encode that input in a low-dimensional space

%~
%\begin{itemize}
%	\item NNs parameterise probability functions
%	\item NNs predict parameters of a probabilistic model
%\end{itemize}

\end{frame}


\begin{frame}{NN as efficient parametrisation}

From a statistical point of view, NNs do not generate data\\
\begin{itemize}
	\item \alert{they parametrise distributions} that \\
	\emph{by assumption} generate data
	\item compact and efficient way to \alert{map from complex side information to parameter space}
\end{itemize}

\vspace{10pt}

\pause
Prediction is done by a decision rule outside the statistical model
\begin{itemize}
	\item e.g. argmax, beam search
\end{itemize}

\end{frame}



\begin{frame}{Maximum likelihood estimation}

Let $p(x|\theta)$ be the probability of an observation $x$\\
~and $\theta$ refer to all of its parameters \\
%~e.g. parameters of NNs involved

~ \pause

Given a dataset $x^{(1)}, \ldots, x^{(N)}$ of i.i.d. observations, \pause
the log-likelihood function gives us a criterion for parameter estimation
\begin{equation*}
\begin{aligned}
\mathcal L(\theta|x^{(1:N)}) &= \pause \log \prod_{s=1}^N p(x^{(s)}|\theta) 
 = \pause \sum_{s=1}^N \log p(x^{(s)}|\theta)
\end{aligned}
\end{equation*} 


\end{frame}

\begin{frame}{MLE via gradient-based optimisation}

If the log-likelihood is {\bf differentiable} and  {\bf tractable}\\
~then backpropagation gives us the gradient
\begin{small}
\begin{equation*}
\begin{aligned}
\grad_\theta \mathcal L(\theta|x^{(1:N)}) &= \pause \grad_\theta \sum_{s=1}^N \log p(x^{(s)}|\theta) 
 &= \pause \sum_{s=1}^N \grad_\theta \log p(x^{(s)}|\theta)
\end{aligned}
\end{equation*}
\end{small}  \pause

and we can update $\theta$ in the direction
\begin{equation*}
\gamma \grad_\theta \mathcal L(\theta|x^{(1:N)})
\end{equation*}
to attain a local maximum of the likelihood function

\end{frame}


\begin{frame}{Big Data}

For large \alert{$N$}, computing the gradient is inconvenient
\begin{small}
\begin{equation*}
\begin{aligned}
\grad_\theta \mathcal L(\theta|x^{(1:N)}) &=   \underbrace{\sum_{s=1}^{\alert{N}} \grad_\theta \log p(x^{(s)}|\theta)}_{\text{too many terms}} \\ \pause
&=   \sum_{s=1}^{\alert{N}} \textcolor{blue}{\frac{1}{ N} } N \grad_\theta \log p(x^{(s)}|\theta) \\ \pause
&= \sum_{s=1}^{\alert{N}}\textcolor{blue}{\mathcal U(s|\sfrac{1}{N})}  N \grad_\theta \log p(x^{(s)}|\theta) \\ \pause
&= \mathbb E_{S\sim \mathcal U(\sfrac{1}{N})}\left[ N \grad_\theta  \log p(x^{(S)}|\theta)\right]
\end{aligned}
\end{equation*} 
\end{small}

$S$ selects data points uniformly at random



\end{frame}

\begin{frame}{Stochastic optimisation}


For large $\alert{N}$, we can use a gradient estimate 
\begin{small}
\begin{equation*}
\begin{aligned}
\grad_\theta \mathcal L(\theta|x^{(1:N)}) 
 &=  \underbrace{\mathbb E_{S\sim \mathcal U(\sfrac{1}{N})}\left[ N \grad_\theta  \log p(x^{(S)}|\theta)\right]}_{\text{expected gradient :)}} \\ \pause
 &\overset{\text{MC}}{\approx} \frac{1}{M} \sum_{m=1}^M N  \grad_\theta \log p(x^{(s_m)}|\theta) \\
 &S_m \sim \mathcal U(\sfrac{1}{N})
\end{aligned}
\end{equation*}
\end{small}  \pause
and take a step in the direction
\begin{small}
\begin{equation*}
\gamma \frac{N}{M} \underbrace{\grad_\theta \mathcal L(\theta|x^{(s_1:s_M)})}_{\text{\alert{stochastic gradient}}}
\end{equation*}
\end{small}
where $x^{(s_1:s_M)}$ is a random mini-batch of size $M$


\end{frame}




\begin{frame}{DL in NLP recipe}


%Vast majority of papers published at ACL

%\begin{small}
%\begin{figure}
%\scalebox{0.8}{
%\begin{tikzpicture}
%\node[obs] (x) {$ x $};
%\node[left=of x] (phi) {$ \phi $};
%\factor[left=of x] {f} {below:$f_w$} {phi} {x} ; 
%%\edge{phi}{x} ;
%\plate {data} {(x)(phi)} {$ N $};
%\end{tikzpicture}
%}
%\end{figure}
%\end{small}
	Maximum likelihood estimation
	\begin{itemize}
		\item  tells you which \alert{loss} to optimise \\
		(i.e. negative log-likelihood)
	\end{itemize}
	
	Automatic differentiation (\emph{backprop})
	\begin{itemize}
		\item ``give me a tractable forward pass and I will give you \alert{gradients}''
	\end{itemize}
	
	Stochastic optimisation powered by backprop
	\begin{itemize}
		\item general purpose gradient-based optimisers
	\end{itemize}

\end{frame}


\begin{frame}{Constraints}

	Differentiability
	\begin{itemize}
		\item intermediate representations must be continuous
		\item activations must be differentiable
	\end{itemize}

	Tractability
	\begin{itemize}
		\item the likelihood function must be evaluated exactly, thus it's required to be tractable
	\end{itemize}


%, any intractable likelihood will leave us in bad territory because
%\begin{itemize}
%	\item stochastic optimisation requires gradient estimates
%	\item which must be unbiased (forget greedy techniques)
%	\item and some estimation techniques are not differentiable (forget MC sampling)
%\end{itemize}

\end{frame}


\section{Latent Variable Models Powered by NNs}
\frame{\tableofcontents[currentsection]}

\begin{frame}{When do we have intractable likelihood?}

{\bf Latent variable models} contain unobserved random variables\\ 
\begin{equation*}
p(x, z|\theta)
\end{equation*}

~ 

thus assessing the marginal likelihood requires \alert{marginalisation of latent variables} 
\begin{equation*}
p(x|\theta) = \int p(x, z|\theta) \dd{z} 
\end{equation*}

\end{frame}





\begin{frame}{Latent variable model}

%Unsupervised problems contain latent random variables\\ 
\begin{columns}
\begin{column}{0.6\textwidth}
Latent random variables
\begin{itemize}
	\item unobserved 
	\item or unobservable
\end{itemize}
\end{column}
\begin{column}{0.2\textwidth}
\begin{tikzpicture}
\node[obs] (x) {$ x $};
\node[latent,above=of x] (z) {$ z $};
\node[right=of x] (theta) {$ \theta $};
\edge{z,theta}{x};

\plate {data} {(z)(x)} {$ N $};
\end{tikzpicture}

\end{column}
\end{columns}

~ \pause


A joint distribution over data and unknowns
\begin{equation*}
p(x, z|\theta) =  p(z) p(x|z, \theta)
\end{equation*}

\end{frame}


\begin{frame}{Examples of latent variable models}

Discrete latent variable, continuous observation
	\begin{small}
	\begin{equation*}
	p(x|\theta) = \underbrace{\sum_{c=1}^K \Cat(c|\pi_1, \ldots, \pi_K) \underbrace{\mathcal N(x|\mu_\theta(c), \sigma_\theta(c)^2)}_{\text{forward pass}}}_{\text{{\bf too many forward passes}}}
	\end{equation*}
	\end{small} 

	\pause
	
Continuous latent variable, discrete observation
	\begin{small}
	\begin{equation*}
	p(x|\theta) = \underbrace{\int \mathcal N(z|0, I) \underbrace{\Cat(x|\pi_\theta(z))}_{\text{forward pass}} \mathrm{d}z }_{\text{\alert{{\bf infinitely many forward passes}}}}
	\end{equation*}
	\end{small}

\end{frame}

\begin{frame}{Intractable gradient}

\begin{small}
\begin{equation*}
\begin{aligned}
\grad_\theta \log p(x|\theta) \pause &= \grad_\theta \log \underbrace{\int p(x, z|\theta) \dd{z}}_{\text{marginal}} \\ \pause
&= \underbrace{\frac{1}{\int p(x, z|\theta) \dd{z}} \int \grad_\theta p(x,z|\theta) \dd{z}}_{\text{chain rule}} \\ \pause
&= \frac{1}{p(x|\theta)} \int \underbrace{p(x,z|\theta) \grad_\theta \log p(x,z|\theta)}_{\text{log-identity for derivatives}} \dd{z} \\ \pause
&= \int \underbrace{\alert{p(z|x, \theta)}}_{\text{posterior}} \grad_\theta \log p(x,z|\theta) \dd{z} %\\ \pause
%&= \mathbb E_{\alert{p(z|x, \theta)}} \left[ \grad_\theta \log p(x,z|\theta) \right]
\end{aligned}
\end{equation*}
\end{small}



\end{frame}

\begin{frame}{Approximations}
	Can we approximate the gradient?
	
	\begin{itemize}
		\item some approximations introduce bias
		\item others break differentiability		
		\item some approximations suffer from both problems
	\end{itemize}
	
	We prefer unbiased approximations. A large bulk of DGM research goes to efficient unbiased gradient estimation.
	
\end{frame}

\begin{frame}{Gradient estimates?}

\begin{equation*}
\begin{aligned}
&\grad_\theta \log p(x|\theta) = \int \alert{p(z|x, \theta)} \grad_\theta \log p(x,z|\theta) \dd{z} \\ \pause
&= \mathbb E_{\alert{p(z|x, \theta)}} \left[ \grad_\theta \log p(x,z|\theta) \right] \\ \pause
&\overset{\text{MC}}{\approx} \frac{1}{K} \sum_{k=1}^K \grad_\theta \log p(x, z_k|\theta) 
\quad \textcolor{gray}{\text{where }z_k \sim p(z|x, \theta)}
\end{aligned}
\end{equation*}

\pause But the posterior is not available!

\begin{equation*}
\begin{aligned}
	p(z|x, \theta) = \pause \frac{p(x, z|\theta)}{\alert{p(x|\theta)}} % = \frac{p(z)p(x|z, \theta)}{\int p(z', x|\theta) \dd z'}
\end{aligned}
\end{equation*}

\end{frame}


\begin{frame}{But why latent variable modelling?}

Some reasons

\begin{itemize}
	\item better handle on statistical assumptions\\
	e.g. breaking marginal independence \pause
	\item organise a massive collection of data\\
	e.g. LDA	 \pause
	\item learn from unlabelled data\\
	e.g. semi-supervised learning \pause
	\item induce discrete representations\\
	e.g. parse trees, dependency graphs, alignments \pause
	%e.g. derivatives are not defined for discontinuous functions
	\item uncertainty quantification\\
	e.g. Bayesian NNs 
\end{itemize}

\end{frame}

\begin{frame}{Examples: Lexical alignment}

Generate a word $x_i$ in L1 from a word $y_{a_i}$ in L2 \pause
	\begin{equation*}
	\begin{aligned}
		p(x|y, \theta) %&= \sum_{a_1=1}^{\abs{y}}\cdots \sum_{a_{\abs{x}}=1}^{\abs{y}} \prod_{i=1}^{\abs{x}} P(a_i|y)P(x_i|y,a_i, a_{<i}) \\
		&\overset{\text{\alert{ind}}}{=} \prod_{i=1}^{\abs{x}} \sum_{a_i=1}^{\abs{y}}\mathcal U(a_i|\sfrac{1}{\abs{y}})p(x_i|y_{\alert{a_i}}) 
		\end{aligned}
	\end{equation*}

\pause

a mixture model whose mixture components are labelled by words \hfill \textcolor{blue}{marginalisation $O(\abs{x}\abs{y})$}

\end{frame}

\begin{frame}{Examples: Rationale extraction}

Sentiment analysis based on a subset of the input \pause
\begin{equation*}
	\begin{aligned}
		p(x|y, \theta) &= \sum_{f_1=0}^{1}\cdots \sum_{f_{\abs{y}}=0}^{1} \left( \prod_{i=1}^{\abs{y}} \Bernoulli(f_i|\theta_{y_i}) \right) p(x|f,y)  \\
		\end{aligned}
	\end{equation*}
where $p(x|f,y)$ conditions on $y_i$ iff $f_i = 1$.

~ \pause

A factor model whose factors are labelled by words \\
\hfill \alert{marginalisation $O(2^{\abs{y}})$}

\end{frame}

\begin{frame}{Examples: Language modelling}

	A (deterministic) RNNLM aways produces the same conditional $p(x_i|x_{<i}, \theta)$ for a given prefix. \pause Isn't it reasonable to expect the conditional to depend on what we are talking about? \pause e.g. \emph{\alert{Rio de Janeiro $\ldots$}}
	\begin{itemize}
		\item history: \emph{\alert{once was the Brazilian capital}}
		\item tourism: \emph{\alert{offers some of Brazil's most iconic landscapes}}
		\item news: \emph{\alert{recently hosted the world cup final}}
	\end{itemize} \pause
	\vspace{-5pt}
	\begin{equation*}
		p(x|\theta) = \int \mathcal N(z|0, I) \prod_{i=1}^{\abs{x}} p(x_i|z, x_{<i}, \theta) \dd z
	\end{equation*}

\end{frame}


\begin{frame}{Deep Generative Models}

Probabilistic models parametrised by neural networks
\begin{itemize}
	\pause
	\item explicit modelling assumptions\\
	one of the reasons why there's so much interest	
	\pause
	\item but requires efficient inference\\
	\pause
	\alert{which is the reason why we are here today}
\end{itemize}

\end{frame}